\documentclass[landscape]{article}
%\usepackage{pslatex}
\usepackage{graphicx}
\usepackage{multicol}
\usepackage{calc}
\usepackage{color}
\usepackage{textcomp}
\usepackage{hyperref}
\usepackage{xspace}
\usepackage[]{url}
\hypersetup{%
  colorlinks=true,  
  linkcolor = black,
  citecolor = black, 
  urlcolor = blue, 
  pdfpagemode=UseNone, 
  pdfstartview=FitH,
  pdfauthor= {Jonas Stein, Randy Andy, Jean-Paul},
  pdftitle= R Reference Card
}

\urlstyle{same}


% Turn off header and footer
\pagestyle{empty}

%\setlength{\leftmargin}{0.75in}
\setlength{\oddsidemargin}{-0.75in}
\setlength{\evensidemargin}{-0.75in}
\setlength{\textwidth}{10.5in}


\setlength{\topmargin}{-0.2in}
\setlength{\textheight}{7.4in}
\setlength{\headheight}{0in}
\setlength{\headsep}{0in}

\pdfpageheight\paperheight
\pdfpagewidth\paperwidth

% Redefine section commands to use less space
\makeatletter
\renewcommand\section{\@startsection{section}{1}{0mm}%
                                     {-24pt}% \@plus -12pt \@minus -6pt}%
                                     {0.5ex}%
                                {\color{black}\normalfont\large\bfseries}}
\makeatother

% Don't print section numbers
\setcounter{secnumdepth}{0}

\setlength{\parindent}{0pt}
\setlength{\parskip}{0pt}

\newcommand{\code}{\texttt}
\newcommand{\bcode}[1]{\texttt{\textbf{\color{blue} #1}}}
\newcommand\F{\code{FALSE}}
\newcommand\T{\code{TRUE}}

%\newcommand{\hangpara}[2]{\hangindent#1\hangafter#2\noindent}
%\newenvironment{hangparas}[2]{\setlength{\parindent}{\z@}\everypar={\hangpara{#1}{#2}}}

\newcommand{\describe}[1]{\begin{description}{#1}\end{description}}
\newcommand{\asroot}{\# }
\newcommand{\asuser}{\$ }
\newcommand{\cp}{{\color{blue}cp}\xspace}
\newcommand{\cpv}{{\color{blue}cpv}\xspace}

% -----------------------------------------------------------------------

\begin{document}

%\raggedright
\footnotesize
\begin{multicols*}{2}

% multicol parameters
% These lengths are set only within the two main columns
%\setlength{\columnseprule}{0.25pt}
\setlength{\premulticols}{1pt}
\setlength{\postmulticols}{1pt}
\setlength{\multicolsep}{1pt}
\setlength{\columnsep}{2pt}

\begin{center}
     {\Large{\textbf{\color{black}Gentoo cheat sheet}}} \\
 \today
\end{center}
\href{mailto:news@jonasstein.de}{Jonas Stein} maintains 
the source since 2013 on \href{https://github.com/jonasstein/gentoo-cheat}{github} feel free to contact him for contributions. A \asroot as first character indicates a command which is usually run in a root shell. 
\everypar={\hangindent=9mm}

\section{Abbreviations}
\cp stands for category + program name like in \code{app-editors/vim}\\
\cpv stands for category + program name + version number like in \code{app-editors/vim-7.3.409}\\

\section{Portage package management}
\bcode{\asroot emerge --sync} download latest package information from the mirror\\
\bcode{\asroot eix-sync} sync the overlay trees \textbf{and} run \code{emerge --sync} in one command\\
\bcode{\asroot emerge -pv \cp} verbose simulation of package installation\\ 
\bcode{\asroot emerge =cpv} install special version of a package\\ 
\bcode{\asroot emerge -C \cp} remove package\\ 
\bcode{\asroot emerge -av --depclean} remove no longer needed packages\\
\bcode{\asroot emerge -uDN world} upgrade the whole system\\
\bcode{\asuser equery b \textacutedbl which dig\textacutedbl} tells which installed package provides dig\\
\bcode{\asroot eix \cp} documentation on a package\\ %\code{packagename} 
\bcode{\asroot porthole} a gui for emerge\\
\bcode{\asroot dispatch-conf} manage config changes after emerge\\
\bcode{\asroot revdep-rebuild} rebuilds packages that depend on an recently updated library\\

\section{USE flags}

\section{Typical administration}
\bcode{\asuser sudo -s} used by a user from sudoers list to get a root shell\\
\bcode{\asroot rc-update add sshd default} Start the ssh daemon in the default runlevel\\
\bcode{\asroot rc-service xdm start} Start the windowmanager now\\
\bcode{\asroot rc-status sshd} request if sshd is running or not\\

\section{Important files}
\bcode{/etc/portage/make.conf} global settings (USE flags, compiler options)\\
\bcode{/etc/portage/package.use} USE flags of individual packages\\
\bcode{/etc/portage/package.keywords} keywords of individual packages (like $\sim{}$amd86 for\\ 64bit-testing)\\
\bcode{/etc/portage/package.license} accepted licenses\\
\bcode{/var/lib/portage/world} list of explicitly installed package atoms

%\section{Installation}
%The official installation is well described in the handbook on \url{http://www.gentoo.org/doc/de/handbook/}. 
%The following steps help me to remember the strategy.\\
%Boot systemrescue-cd and mount the target root partition to \code{/mnt/gentoo}. (
\end{multicols*}
\end{document}
